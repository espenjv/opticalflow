\chapter{The Isotropic Flow-Driven Method}
One drawback of the image-driven approach to regularization is that there is often a great deal of oversegmentation, since image boundaries are not necessarily flow boundaries. The solution is to decrease the smoothing on flow boundaries, a so called flow-driven approach. But an obvious problem is that the flow boundaries are not known a priori.
\section{Subquadratic Penalization}
Shulman and Herve \cite{ShulmanHerve} suggested using a subquadratic penalizer instead of a quadratic one, arguing that a quadratic penalizer assumes a Gaussian distribution of the flow gradients which would penalize large gradients, assumed to correspond to flow boundaries, too much. The smoothness term can be written as
\begin{align*}
V_{IF}(\nabla u, \nabla v) &= \psi_V \left( |\nabla u|^2 + |\nabla v|^2 \right) \\ 
&= \psi_V \left( u_{\textbf{s}_1}^2 + u_{\textbf{s}_2}^2 + v_{\textbf{s}_1}^2 + v_{\textbf{s}_2}^2 \right),
\end{align*} 
where $\psi_V(s^2)$ is some subquadratic penalizer function performing a nonlinear isotropic diffusion. The contribution to (\ref{EL}) is $\nabla \cdot (\partial_{u_x} V, \partial_{u_y} V)$. Computing the individual components, we get
\begin{align*}
\partial_{u_x} V = 2 \psi_V' \left( u_{\textbf{s}_1}^2 + u_{\textbf{s}_2}^2 + v_{\textbf{s}_1}^2 + v_{\textbf{s}_2}^2 \right) u_x \\
\partial_{u_y} V = 2 \psi_V' \left( u_{\textbf{s}_1}^2 + u_{\textbf{s}_2}^2 + v_{\textbf{s}_1}^2 + v_{\textbf{s}_2}^2 \right) u_y,
\end{align*}
and equivalent for $(\partial_{v_x} V, \partial_{v_y} V)$. Thus the diffusion matrix of (\ref{EL_regu}) is given as
\begin{align*}
\Theta_{IF} = 2 \psi_V'\left( u_{\textbf{s}_1}^2 + u_{\textbf{s}_2}^2 + v_{\textbf{s}_1}^2 + v_{\textbf{s}_2}^2 \right) I,
\end{align*}
where $I$ is the identity matrix, which is seen to be a function of the flow gradient. As a convex penaliser, Cohen (1993) suggested the following total variation regulariser:
\begin{align}
\label{TV_regu}
\psi_V(s^2) = \sqrt{s^2 + \epsilon^2},
\end{align} 
which gives
\begin{align*}
\psi_V'(s^2) = \frac{1}{2 \sqrt{s^2 + \epsilon^2}}.
\end{align*}
This penaliser function results in the following Euler-Lagrange system
\begin{equation}
\begin{aligned}
\label{EL_LD}
\partial_u M - \frac{1}{\sigma^2} \left(\frac{\partial}{\partial x}\left[ \frac{u_x}{\sqrt{|\nabla u|^2 + |\nabla v|^2  + \epsilon^2}} \right] + \frac{\partial}{\partial y} \left[ \frac{u_y}{\sqrt{|\nabla u|^2 + |\nabla v|^2  + \epsilon^2}} \right] \right) = 0 \\
\partial_v M - \frac{1}{\sigma^2} \left(\frac{\partial}{\partial x}\left[ \frac{v_x}{\sqrt{|\nabla u|^2 + |\nabla v|^2 + \epsilon^2}} \right] + \frac{\partial}{\partial y} \left[ \frac{v_y}{\sqrt{|\nabla u|^2 + |\nabla v|^2 + \epsilon^2}} \right] \right) = 0,
\end{aligned}
\end{equation}
which is clearly a non-linear system. (Something more about TV regularizers? Rudin et al.)


\subsection{The Lagged Diffusivity Fixed Point Method}
The non-linear system (\ref{EL_LD}) is can be solved using the method of lagged diffusivity. Chan and Mulet \cite{Chan1999} showed that the lagged diffusivity fixed point method used in TV restoration of images can be traced back to Weiszfeld's method of minimizing Euclidean lengths, and for which global and linear convergence can be shown. From (\ref{EL_LD}) we can define the fixed point method as
\begin{align*}
\partial_{u^{k+1}} M - \frac{1}{\sigma^2} \left(\frac{\partial}{\partial x}\left[ \frac{u^{k+1}_x}{\sqrt{|\nabla u^k|^2 + |\nabla v^k|^2 + \epsilon^2}} \right] + \frac{\partial}{\partial y} \left[ \frac{u^{k+1}_y}{\sqrt{|\nabla u^k|^2 + |\nabla v^k|^2 + \epsilon^2}} \right] \right) &= 0 \\
\partial_{v^{k+1}} M - \frac{1}{\sigma^2} \left(\frac{\partial}{\partial x}\left[ \frac{v^{k+1}_x}{\sqrt{|\nabla u^k|^2 + |\nabla v^k|^2 + \epsilon^2}} \right] + \frac{\partial}{\partial y} \left[ \frac{v^{k+1}_y}{\sqrt{|\nabla u^k|^2 + |\nabla v^k|^2 + \epsilon^2}} \right] \right) &= 0,
\end{align*}
where we solve for $u^{k+1}$ and $v^{k+1}$ given $u^k$ and $v^k$ from the previous iteration. This is a linear system in each iteration, and it can be solved in the same manner as the previous methods. The system to be solved at iteration $k+1$ is
\begin{align*}
(D^T D + \frac{1}{\sigma^2} L^T\Theta^{k}L) \textbf{w}(\xi)^{k+1} = - D^T \textbf{c}(\xi),
\end{align*}
for $\xi \in \Omega$, where $\Theta^k$ is the following 4mn-by-4mn diagonal block matrix
\begin{align*}
\Theta^k = \left[
\begin{array}{c|c|c|c}
\chi^k & 0 & 0 & 0 \\ \hline
0 & \chi^k & 0 & 0 \\ \hline
0 & 0 & \chi^k & 0 \\ \hline
0 & 0 & 0 & \chi^k
\end{array}
\right].
\end{align*}
The submatrix $\chi^k$ is the mn-by-mn diagonal matrix with the values 
\begin{align*}
\chi^k_i &= 2 \psi_V'\left( |\nabla \textbf{w}(\xi^i)|^2 \right) \\
& = \frac{1}{\sqrt{|\textbf{w}(\xi^i)|^2 + \epsilon^2}}
\end{align*}
along its diagonal.

\subsection{Results for Isotropic flow driven}
....

