\chapter{Introduction}
When an observer of an object moves relative to the object, there is an apparent relative motion in the image plane of the observer. The problem of determining this relative motion from a sequence of images is called the Optical Flow problem. The analysis is not so much dependent on prior knowledge of the scene, but on the image sequence itself. This independency makes it applicable in many different fields. More concisely, one wants to find flow vector components $u,v \in \mathbb{R}$ by looking at the change in brightness $f(x,t)$ at a specific pixel from one frame to another for $x \in \Omega$, where $\Omega$ is considered to be a rectangular domain. This is problematic because we can not independently determine a vector of 2 components using one constraint coming from the change in image brightness at a point $x \in \Omega$. Thus we need to impose other constraints to make the problem solvable.

Is there always a relationship between the change in the brightness and the movement of objects in the image? It is not hard to see that the answer is no. For instance, imagine rotating a uniform sphere exhibiting a nonuniform brightness pattern over its surface. This rotation is not observable in the image plane, and would result in zero optical flow. Also, if the illumination of the image scene changes rapidly, brightness changes in the image plane may not be due to moving objects. These examples illustrate that optical flow does not always correspond to the relative movement of an object. Nonetheless, in the following model for optical flow the image scene is assumed to be simple so that brightness changes can be directly related to object motion.    